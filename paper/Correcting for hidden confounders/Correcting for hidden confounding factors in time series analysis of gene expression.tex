% !TEX encoding = UTF-8 Unicode
\documentclass[11pt,a4paper,titlepage,twoside,tablecaptionabove]{article}

% Schriftart:
\usepackage[T1]{fontenc}
\usepackage[utf8]{inputenc}
\usepackage{palatino}

\usepackage{MnSymbol, xcolor}
\usepackage{amsmath}
\usepackage{graphicx}

\usepackage{geometry}
\geometry{a4paper,left=30mm,right=35mm, top=40mm, bottom=40mm}

% section styles
\newcommand{\changefont}[3]{
  \fontfamily{#1}\fontseries{#2}\fontshape{#3}\selectfont}
%\newcommand\bigheading{\changefont{cmbr}{bx}{n}}
%\newcommand\littleheading{\changefont{cmbr}{b}{n}}
\newcommand\TODO[1]{{\sc \color{red}{Todo: #1}}}

%\usepackage{sectsty}
%\chapterfont{\huge\bigheading\renewcommand\it{\changefont{pplx}{bx}{it}}}
%\sectionfont{\Large\littleheading\renewcommand\it{\changefont{pplx}{bx}{it}}}
%\subsectionfont{\large\littleheading\renewcommand\it{\changefont{pplx}{bx}{it}}}
%\subsubsectionfont{\littleheading}

% spacing
\usepackage{setspace}

\definecolor{lightgray}{rgb}{.0,.0,.0} \usepackage[%
colorlinks, citecolor=lightgray, linkcolor=lightgray,%
filecolor=lightgray, pagecolor=lightgray,%
urlcolor=lightgray,breaklinks]{hyperref}

% Caption Settings:
\usepackage[margin=10pt,font=small,labelfont=bf,hang]{caption}
\DeclareCaptionLabelFormat{bez}{#2}

\begin{document}

\begin{titlepage}
  \changefont{cmbr}{m}{n}
  \begin{center}
    {\LARGE Max-Planck-Institute T\"ubingen}\\[5cm]
    {\huge \bfseries{Correcting for hidden confounding factors in time
        series analysis of gene expression\\[1.5cm]}}
    {\large Oliver Stegle, Max Zwießele, Nicol\'o Fusi}\\[0.5cm]
    \today\\[3cm]
  \end{center}
\end{titlepage}

\begin{abstract}
  Differential gene expression time series studies facilitate immense
  insights into gene activity and genetic interaction. It studies
  differences of gene expression in the course of time of two or more
  treatments of an organism. Often there is no evidence for wether a
  set of experiments fundamentally differ in their expression over
  time, or results are consequences of differing experimental methods
  and technical conditions. There are a variety of normalizing
  methods, which try to solve technical and biological confounding
  factors prior to analyses. We invent a new way of using Gaussian
  Process Latent Variable Models (GPLVM) to correct for intra and
  inter experiment hidden latent variable confounding factors of time
  series experiments. GPLVM is a statistical method based on Gaussian
  Processes to find key features in a multidimensional dataset. As
  this method can not only detect these features but report their
  significance, we are able to correct for common confounding factors
  over arbitrary dimensions.  We examined a dataset \TODO{dataset} to
  find and eliminate such confounding factors, correcting for hidden
  multidimensional features. Our method can find \TODO{result}.
\end{abstract}

\section{Introduction}
\label{sec:introductions}

In gene expression time series experiments (see a review
\cite{bar2004analyzing}) temporal changes in gene expression are
measured. These temporal changes lead to distinguish pathogenic
responses of an organism to a particular treatment. One usually has
two experiments at a time, one untreated, mostly healthy control and
one treatment. The organism is treated by one (or more) pathogenic
substances and in continuous time steps their gene expression is
measured. Each experiment has to be normalized within and between
control and treatment to provide comparable results. Many normalizing
methods tackle these problems using house-keeping genes as an anchor
for other time series to fit onto
\cite{vandesompele2002accurate,aach2001aligning} or building
statistical models for variation in gene expression measurement errors
\cite{rocke2001model}. However, there is no evidence for these methods
to correct for inter experimental confounders, which are not based on
one particular run of gene expression measurement. Thus, correcting
for hidden confounders in gene expression time series can provide a
huge boost in accuracy of differential gene expression detection. We
believe these hidden confounders to arrise due to experiments being
done by different experimenters, in different laboratories or even at
different day times.

Gene expression time series experiments can give insights into genetic
inter relationships of {\it cis} and {\it trans} regulatory
effects. Inferring these effects lead to transcriptional gene
regulatory networks describing the temporal effects of genetic
variation caused by the pathogenic substances the organism is treated
with. Clusters of differentially expressed genes regulated similarly
(in time and pattern) give rise to detecting {\it trans} regulatory
effects \cite{stegle2010robust}. In contrast to detecting time
dependent relationships in differentially expressed genes, we detect
common confounding factors using probabilistic principal component
analysis modeled by gaussian process latent variable models (GPLVM)
\cite{lawrence2004gaussian}. This probabilistic approach to detect low
dimensional significant features can be interpreted as detecting
common confounding factors in time series experiments by applying
GPLVM in advance to two-sample tests \cite{stegle2010robust} on the
whole dataset. Two-sample tests on Gaussian Processes decide
differential expression as the bayes factor of marginal probabilities
for control and treatment being modeled by one common or two separate
underlying function(s). As GPLVM is based on Gaussian Processes it
provides a covariance structure of confounders in the dataset. We take
this covariance structure between features to build up a two-sample
Gaussian Process model taking confounding factors throughout the
dataset into account.

\section{Building a Gaussian Process model taking confounding factors
  into account}
\label{sec:build-gauss-proc}

In probabilistic models for gene expression time series we are given
for $N$ genes $D$-dimensional, centered expression levels $\{{\bf
  y}_n\}_{n=1}^N$ for each input (series of time points) ${\bf
  x}_n$. Under the Gaussian Process model \cite{rasmussen2004gaussian}
we write the distribution over gene expression time series as
\begin{equation}
  \label{eq:1}
  p({\bf Y}|{\bf X},\beta) = \mathcal N(0, {\bf K_\beta})\enspace,
\end{equation}
where ${\bf X,Y}$ are concatenations of ${\bf x}_n$ and $\{{\bf
  y}_n\}_{n=1}^N$, respectively and ${\bf K}$ is the covariance matrix
of all pairwise covariances; ${\bf K}_{i,j} = k({\bf x}_i, {\bf
  x}_j;\beta)$.

In GPLVM, instead of learning the hyperparamters $\beta$ a subset of
the inputs ${\bf X}$, referred to as the latent space, are learned and
reported as ${\bf X}_{i=1}^q$. With this setting learning $\bf X$ can
be written as
\begin{equation}
  \label{eq:2}
  {\bf X} = {\bf U}_q{\bf LV}^T\enspace,
\end{equation}
which leads the learned ${\bf X}$ to be the probabilistic
interpretation of the $q$ most significant principal components
\cite{lawrence2004gaussian}. Using this interpretation, we learn ${\bf
  X}_{\text{GPLVM}}$, further called confounders, on the whole dataset
with an arbitrary chosen $q$ by GPLVM to use it in the two-sample
model as correction in covariance. The model for two-sample comparison
may then be written as
\begin{equation}
  \label{eq:3}
  p({\bf Y}|{\bf X},\theta) = \mathcal N(0, {\bf K_\theta} +
  \alpha {\bf X_\text{GPLVM}}\bf X_\text{GPLVM}^T\enspace, 
\end{equation}
${\bf K}_\theta$ is again the covariance matrix between each ${\bf
  x}_n$, respectively, parameterized by $\theta$.

To be able to decide wether a gene is differentially expressed or not,
we build a score based on the Bayes Factor on two models. One model
assumes treatment and control to come from one underlying function and
the other assumes one underlying function for treatment and control,
respectively. With that the Bayes Factor score can be written as
\begin{equation}
  \label{eq:4}
  \text{BF-Score}: \log\frac{p({\bf Y}^\text{t}|{\bf
      X}^\text{t},\theta^\text{t})\cdot p({\bf Y}^\text{c}|{\bf X}^\text{c},\theta^\text{c})}{p({\bf Y}|{\bf X},\theta)}\enspace,
\end{equation}
where t and c indicate only the part of data set which correspond to
treatment and control, respectively. Thus, if positive this score
indicates at one function for each condition, whereas if negative it
is more probable that both conditions come from the same underlying
function.

\subsection{Confounder learning}
\label{sec:confounder-learning}

Learning confounders ${\bf X}_\text{GPLVM}$ incorporates a covariance
function into the probabilistic interpretation of PCA. We use the
structure of different covariance functions to explicitly model
structure of gene expression time series. The first and most simple
covariance function is the linear one (see
\ref{sec:line-covar-conf}). Another covariance function takes the time
series aspect into account, while yet another one even goes a step
further, modeling condition dependent time patterns explicitly.

\subsubsection{Linear covariance for confounders}
\label{sec:line-covar-conf}

The linear covariance function is
\begin{equation}
  \label{eq:5}
  k_\text{lin}({\bf X},{\bf X'}) = \alpha {\bf X}{\bf X'}^T\enspace,
\end{equation}
such that the contribution of principal components comes from the
whole dataset linearly. These components should reflect the fact that
we are using two separate conditions in at least one of the $q$
confounders. 

% \subsubsection{Linear confounder covariance with explicit time model}
% \label{sec:line-conf-covar}

% Studying time series data we can make use of the fact that the data is
% well structured in input space, namely it is time. With this knowledge
% we are able to incorporate time into covariance structure, which
% explains one common attribute in data, the linearity of time. For
% simplicity we write the time model dependent on the time line $\bf
% t$. With that we can write the covariance structure for explicit time
% modeling in the following way. We use a squared exponential covariance
% function for modeling the time, such that the time points are more dependent
% on each other, if nearby and vice versa:
% \begin{align}
%   \label{eq:6}
%   k_t(t,t') & =
%   A\cdot\exp\left(-\frac{||t-t'||^2}{2L^2}\right)\enspace,\text{
%     with}\\
%   \{{\bf K}_{\bf t,t'}\}_{i,j} & = k_t(t_i,t'_j)
% \end{align}
% where $A$ and $L$ are hyperparameters explaining the magnitude and
% smoothness of the underlying function, respectively. In the setting of
% learning confounders we write the covariance for confounders
% \begin{equation}
%   \label{eq:7}
%   k_\text{time}({\bf X},{\bf X'},{\bf t,t'}) = k_\text{lin}({\bf
%     X,X'}) + {\bf K_{\bf t,t'}}
% \end{equation}

\subsubsection{Linear confounders taking time and conditional data structure into account}
\label{sec:line-conf-taking}

As stated above we are working with time series comparing two
conditions to each other, i.e treatment and control. We make use of
this knowledge by incorporating this structure into the covariance structure for 
confounders. We denote the conditions by an indicator vector $\bf c$ with
\begin{equation}
  \label{eq:8}
  c_i =
  \begin{cases}
    0,\quad\text{ if }{\bf y}_i \text{ corresponds to the control group}\\
    1,\quad\text{ otherwise.}
  \end{cases}
\end{equation}
We define a covariance based on a kronecker delta comparison between
conditions, such that the time based covariance is only incorporated
if the compared inputs correspond to the same condition:
\begin{equation}
  \label{eq:10}
  k_\delta(c,c') = \delta_{c,c'}
\end{equation}
This means that the confounders are only learned for non conditional
and time dependent connections between the inputs.
We can incorporate the conditions into the covariance
function of learning confounders by
\begin{equation}
  \label{eq:9}
  k_\text{cond}({\bf X,X',t,t',c,c'}) = k_\text{lin}({\bf
    X,X'}) + {\bf K_{\bf t,t'}} \cdot {\bf K_{\bf c,c'}}\enspace,
\end{equation}
where $\{ {\bf K_{c,c'}} \}_{i,j}=k_\delta(c_i,c_j)$.
With this covariance structure we avoid learning time and condition
dependent confounders.

\section{Results}
\label{sec:results}

We apply two-sample tests on a simulated dataset. We used a ground
truth to generate ROC plots of the result. We simulated confounders from a
Gaussian distribution with linear covariance \eqref{eq:5}. We compared
results from three different approaches. One is to run a standard
Gaussian Process, without learning confounders, second is to
put in the simulated confounder covariance as fixed and third learning
the confounders. We compare linear covariance model with the model
taking time and conditional data structure into account (called 'time
x linear covariance'). See figure \ref{fig:ROC_linear} for the
resulting ROC curves. 

\begin{figure}[tps]
  \centering
  \includegraphics[width=.7\textwidth]{images/linear covariance learned confounders as covariance.png}
  \caption{ROC plot of two-sample test taking confounders into
    account. linear covariance corresponds to \eqref{eq:5} as
    covariance, time x linear corresponds to \eqref{eq:9}. standard is
    the result for applying standard Gaussian Process to the
    confounded data.\label{fig:ROC_linear}}
\end{figure}

\bibliographystyle{alpha} \bibliography{../../../../../Dropbox/Paper/Master.bib}
% \addcontentsline{toc}{chapter}{References}

\end{document}
